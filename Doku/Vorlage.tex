\documentclass[11pt, oneside, halfparskip, smallheadings, automark]{scrreprt}

\usepackage[utf8]{inputenc}
\usepackage[T1]{fontenc}
\usepackage[ngerman]{babel}
\usepackage{palatino}

% Weiter Pakete können hinzugefügt werden
\usepackage{
  booktabs,
  graphicx,
  xcolor
}



%%  Für klickbare Inhaltsverzeichnise, Verweise etc.
\usepackage[
  bookmarks=true,           % Lesezeichen erzeugen
  bookmarksopen=true,
	colorlinks=true,
	linkcolor=black, 
	anchorcolor=black, 
	citecolor=black, 
	filecolor=black, 
	menucolor=black, 
	pagecolor=black,
	urlcolor=black
]{hyperref}

%%  Abkürzungsverzeichnis
\usepackage[
     footnote,
     nohyperlinks
]{acronym}

%% Neue Makros
\newcommand{\infoprojekt}{Informatikprojekt Klasse 10}

%% Seiten- und Layouteinstellung
\usepackage[left=3.3cm,%
            top=3cm,%
            bottom=3cm,%
            textwidth=15.5cm,%
            marginparwidth=0cm]{geometry}

\usepackage{setspace}
\onehalfspacing					%%  1,5 zeilig

 
%% Für Bildunterschriften etc.
\usepackage{array}
\usepackage[font=small, labelfont=scriptsize,sf,textfont=it]{caption}      

%% Kopf- und Fußzeilen
\usepackage{scrpage2}
\clearscrheadings
\clearscrplain
\pagestyle{scrheadings}
\rohead{\rightmark}
\rofoot[\pagemark]{\pagemark}
\setheadsepline{.4pt} % Linie unter dem Head

%% Fussnoten bündig
\deffootnote{1.2em}{0em}{%
            \makebox[1.2em][l]{\bfseries\thefootnotemark}}


%   ********************************************************************************
%
%        Anpassen, oder entwerfen einer eigenen Titelseite

\title{Vorlage zum Erstellen einer Dokumentation}
\author{Max Mustermann \\ Moritz Mustermann}

\subject{\infoprojekt}
\publishers{\emph{betreut durch:} \\ Betreuer A \\ Betreuer B}

%
%   ********************************************************************************


\usepackage{blindtext}
\begin{document}

\maketitle

\tableofcontents


%%  Abkürzungsverzeichnis

\chapter*{Abkürzungsverzeichnis}

\begin{acronym}
 \acro{CD}{Compact Disk}
 \acro{DVD}{Digital Video Disc}
\end{acronym}
\chapter{Einleitung}
\blindtext

\blindtext 

Die \acfi{DVD} ist ein digitales Speichermedium. Die \acl{DVD} zählt zu den optischen Speichermedien, wie auch die \ac{CD}.

\chapter{Theoretische Grundlagen}

% Hier ist was, das wahrscheinlich schon zu praktisch ist, aber wir benutzen keine phys. Formeln o.ä.
% Dashalb regex, ajax, json

% ja, hast recht. kannst einfach noch das machen:
% - Server-Client-Modell
% 	- Aufgaben (siehe server.tex)
%	- Kommunikation (HTTP, ajax)
% 		- Upload
%		- Templates
%		- Static Assets
% - Datenspeicherung (json, xml)
% - Datenkonvertierung (regex)
% - Websicherheit (SSL, hashes)

% Aber wirklich, kann alles ganz kurz gefasst sein.

Ein wichtiges Prinzip beim Generieren der Website ist das Umwandeln des \texttt{XML}-Codes in ein
für Template-Dateien benutzbares \texttt{JSON}-Format. Wir haben uns dazu entschieden, für diesen
Vorgang regExp zu benutzen: Ein Platzhalterstring wird auf eine Quelle gematcht, d.h. alle Strings
mit dem entsprechenden Muster werden unter Schlüsseln zurückgegeben. So wird der reguläre Ausdruck
\url{/^[ \t]*\$/} auf alle nur mit Leerzeichen und Tabs gefüllten Zeilen passen. Damit lassen sich
verschachtelte Klassenangaben wie bsp. \texttt{10A/ 10ABCET1} auftrennen: die Ethikgruppe 1 der
Klassen 10a, b und c ist hier gemeint. Um die Daten letztendlich in einen vom Client darstellbaren
(X-)HTML-Code umzuwandeln benutzen wir den Template-Parser von Pyramid. Dabei werden in einem Html-
Dokument Angaben gemacht, die vom Server mit entsprechenden Informationen ersetzt werden. Dabei wird
das zuvor generierte JSON- Dokumentverwendet. Dieses ist ein weit verbreitetes Speicherformat,
welches Daten in Schlüsseln und Werten dartellt, die durch Listen und Bereiche strukturiert sind.
Die ursprüngliche Quelle für dieses Dokument liegt hier bei einem Stundenplanverwaltungs-programm
der Schule. Die dort generierten XML- Dateien werden über die \url{/upload} seite hochgeladen. Dafür
werden jQuery und AJaX verwendet. jQuery ist eine beliebte JavaScript Bibliothek, die dem Anwender
viel Arbeit wie das manuelle Registrieren von Listenern erspart. AJaX steht für \glqq Asynchronous
JavaScript and XML\grqq und bedeuted, dass über JS die Seite mit dem Server kommunizieren kann, ohne
eine komplette HTML-Seite als Rahmen zu übertragen. Damit können sämtliche Dateien problemlos
übertragen werden.
\chapter{Programmdokumentation}
\blindtext

\blindtext

\blindtext

\chapter{Programmablaufprotokolle}

\blindtext

\blindtext

\blindtext

\input{texte/zusammenfassung.tex}

% Glossar

\chapter*{Glossar}

\begin{description}
\item [blindtext] Blindtext ist ein \LaTeX-Paket, welches es zu Demonstrationszwecken erlaubt, an beliebigen Stellen eines Dokumentes Fließtext ein zu setzen. Es handelt sich immer um den folgenden Text:

\blindtext
\item [blindtext] Blindtext ist ein \LaTeX-Paket, welches es zu Demonstrationszwecken erlaubt, an beliebigen Stellen eines Dokumentes Fließtext ein zu setzen.
\end{description}
\bibliography{litlist}
\bibliographystyle{amsalpha}
\begin{thebibliography}{999}

\bibitem{Stackoverflow1} Stackoverflow: \textit{Cookie Management}\\ 
			{\small WWW, Stand: 29.05.2016,}\\ 			
			\url{http://stackoverflow.com/questions/9090912/handling-javascript-cookies}
\bibitem{CSSTricks} CSS-Tricks: \textit{Flexbox-Design}\\ 
			{\small WWW, Stand: 29.05.2016,}\\ 			
			\url{https://css-tricks.com/snippets/css/a-guide-to-flexbox/}
\bibitem{Pyramid} Pyramid Webframework: \textit{Referenz und Dokumentation}\\ 
			{\small WWW, Stand: 29.05.2016,}\\ 			
			\url{http://docs.pylonsproject.org/en/latest/docs/pyramid.html}
\bibitem{W3Schools} W3schools: \textit{CSS-Transformationen}\\ 
			{\small WWW, Stand: 29.05.2016,}\\ 			
			\url{http://www.w3schools.com/css/css3_2dtransforms.asp}
\bibitem{JSON} Python: \textit{JSON Dekodieren}\\ 
			{\small WWW, Stand: 29.05.2016,}\\ 			
			\url{https://docs.python.org/2/library/json.html}

	
%	Vorname, Nachname, (erscheinungsjahr), Titel(aus(/title)), 
%	Art(www.seite:stand: letze Änderung, url, (Zugriff: Datum, Uhrzeit).
\end{thebibliography}


\listoffigures
\listoftables

\chapter*{Nutzungserklärung}
Hiermit erklären wir uns einverstanden, dass die Dokumentation und das Projekt für
schulische Zwecke verwendet werden dürfen.
Wir gestatten, die Dokumentation und das Projekt auf der Website der Schule und zum
Tag der offenen Tür zu veröffentlichen.

\chapter*{Erklärung der eigenständigen Erstellung des Projektes}
Hiermit versichern wir, dass wir die Dokumentation und Projekt selbstständig verfasst und
keine anderen als die angegebenen Quellen und Hilfsmittel benutzt haben sowie das alle
Ausführungen, die den Quellen wörtlich oder sinngemäß entnommen wurden, als solche
gekennzeichnet wurden.

\end{document}
