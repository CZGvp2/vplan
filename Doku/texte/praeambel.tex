

%%  Für klickbare Inhaltsverzeichnise, Verweise etc.
\usepackage[
  bookmarks=true,           % Lesezeichen erzeugen
  bookmarksopen=true,
	colorlinks=true,
	linkcolor=black, 
	anchorcolor=black, 
	citecolor=black, 
	filecolor=black, 
	menucolor=black, 
	pagecolor=black,
	urlcolor=black
]{hyperref}

%%  Abkürzungsverzeichnis
\usepackage[
     footnote,
     nohyperlinks
]{acronym}

%% Neue Makros
\newcommand{\infoprojekt}{Informatikprojekt Klasse 10}

%% Seiten- und Layouteinstellung
\usepackage[left=3.3cm,%
            top=3cm,%
            bottom=3cm,%
            textwidth=15.5cm,%
            marginparwidth=0cm]{geometry}

\usepackage{setspace}
\onehalfspacing					%%  1,5 zeilig

 
%% Für Bildunterschriften etc.
\usepackage{array}
\usepackage[font=small, labelfont=scriptsize,sf,textfont=it]{caption}      

%% Kopf- und Fußzeilen
\usepackage{scrpage2}
\clearscrheadings
\clearscrplain
\pagestyle{scrheadings}
\rohead{\rightmark}
\rofoot[\pagemark]{\pagemark}
\setheadsepline{.4pt} % Linie unter dem Head

%% Fussnoten bündig
\deffootnote{1.2em}{0em}{%
            \makebox[1.2em][l]{\bfseries\thefootnotemark}}
