\section{Client}
Ein Client ist ein Programm oder Gerät, dass auf einer Seite eines Netwerkes mit einem Browser auf der anderen kommuniziert. In diesem Fall ist dies ein Browser eines Besuchers der Seite. Das Gerät des Benutzers kann sowohl ein normaler PC als auch ein Smartphone oder ein Tablet sein. 
\subsection{Funktion}
Die clientseitige Anwendung, also von der Software des Benutzers ausgeführte Programmelemente des Vertretungsplans, muss viele Funktionen liefern und sowohl mit dem Benutzer als auch mit dem Server kommunizieren. Darunter fallen die eigentliche Darstellung der Seite, aber auch Hintergrundaufgaben wie die Verwaltung von Datenpaketen. Einen relativ großen Teil dieser machen recht simple Aufgaben aus, für die eine Kommunikation mit dem Server zu zeitaufwändig wäre. Während clientseitig laufende Programme eine Verbindung mit dem Server auch ohne Neuladen der Seite möglich machen, erfolgen viele Zugriffe mit einer gedrosselten Mobilverbindung, was den Nutzen der Seite durch große Wartezeiten erheblich einschränken würde.

\subsection{Darstellung der Seite}
Die größte Schwierigkeit in diesem Gebiet ist die Diversität der Clientgeräte. Verschiedenste Bildschirmauflösungen, Seitenverhältnisse und beispielsweise Standardschriftgrößen fordern von einer Webseite viel Flexibilität. Bei Smartphones ist es wichtig, dass der Inhalt bedingt durch die geringere Bildschirmgröße nur dann gut zu erkennen ist, wenn er entsprechend skaliert wird. Notwendig is auch, dass Menüs, Schaltflächen und Links für die Bedienung durch ein Touchinterface geeignet sind. Bei einer Desktopseite kann mehr freier Raum gelassen werden, dessen Nutzung bei kleineren Displays notwendig ist, um dennoch eine gewisse Informationsdichte aufrecht zu erhalten. Aus Gründen der Sicherheit und Privatsphäre kann ein Server aber nicht vor der eigentlichen Kommunikation zwischen diesen Gerätetypen unterscheiden. Während es theoretisch möglich ist, nach dem Herstellen einer Verbindung über \texttt{AJaX} Daten nachzuladen, ist es praktischer wenn die Seite eigenständig bestimmen kann, um welchen Gerätetyp es sich handelt. Dementsprechend ist eine Reaktion und Anpassung des Seitenlayouts nach dem Übertragen der Daten durch das Ausführen von Code durch den Client notwendig.

\subsection{Benutzereingaben}
Ein weiterer Schwerpunkt der Clientanwendung ist das Reagieren auf Benuterarbeiten. So müssen Menüs geöffnet und geschlossen werden, Informationen ein- und ausgeblendet, Anfragen weitergeleitet und auch Dateien versendet werden. Viele kleine Codeabschnitte, die unterschiedlichste Aufgaben erfüllen, nehmen insgesamt einen sehr großen Anteil des Programmcodes ein.

\subsection{Umsetzung}
Das Gerüst einer Seite wird von \HTML\ dargestellt. Um eine Seite bei der Benutzung dynamisch und Serverunabhängig zu machen, werden Stil- und Skriptsprachen verwendet. Diese sind in fast allen Fällen \texttt{CSS} und \texttt{JavaScript}. Oft werden in Zusammenarbeit mit diesen auch Bibliotheken wie \texttt{Three.js} oder in unserem Fall \texttt{jQuery} verwendet.\\
Skripte und Stildateien sind im \HTML\ velinkt und werden vom Browser direkt nach diesem angefordert. Dann beginnt ein \JavaScript\ Interpreter das Programm durchzuarbeiten, wobei einige Listener vorbereitet werden. Dem Client wird also mitgeteilt, dass bestimmte Funktionen bei bestimmten Ereignissen aufzurufen sind. Ein Beispiel hierfür findet sich beim Wechsel zwischen zwei verschiedenen Tagen, bei dem ein Klick mehrere Abläufe wie das Starten einer Animation und das Ändern von Beschriftungen hervorruft.\\
Die komplette Darstellung der Seite, unter anderem auch die Animationen, werden durch die Stilsprache \CSS\ festgelegt, die in vielen Gebieten auch außerhalb des Webdesigns Anwendung findet. \CSS\ steht dabei für \texttt{Cascading Stylesheets}, übersetzt etwa \glqq Gestufte Gestaltungsbögen\grqq.
