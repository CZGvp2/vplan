\section{Client}
Der Client ist in diesem Fall ein Browser eines Besuchers der Seite. Das Gerät des Benutzers kann sowohl ein normaler PC als auch ein Smartphone oder ein Tablet sein. Aus Gründen der Sicherheit und Privatsphäre kann ein Server nicht vor der eigentlichen Kommunikation zwischen diesen Gerätetypen unterscheiden. Während es theoretisch möglich ist, nach dem Herstellen einer Verbindung über \texttt{AJaX} Daten nachzuladen ist es praktischer, wenn die Seite eigenständig bestimmen kann, um welchen Gerätetyp es sich handelt. Dementsprechend ist eine Reaktion und Anpassung des Seitenlayouts nach dem Übertragen der Daten durch das Ausführen von Code durch den Client notwendig.\\

Außerdem muss die Seite ohne Serverunterstützung sämtliche Aufgaben durchführen können, für die eine erneute Datenübertragung schlichtweg zu Zeitaufwändig wäre. Beispiele hierfür sind das Filtern nach bestimmten Lehrern bzw. Klassen, das Anzeigen von Menüs, das Wechseln zwischen verschiedenen Tagen etc.\\

Um eine Seite dynamisch und unabhängig zu machen, werden Stil- und Skriptsprachen verwendet. Die gebräuchlichen sind \texttt{CSS} und \texttt{JavaScript}.

% Das hier ist halbwegs fertig, aber nicht ganz und deswegen noch nicht hier
\section{Design}
%%
\subsection{Tabellendarstellung}
%
\subsection{Slides}
%


\section{Interaktivität}
%%
\subsection{Filter}
%
\subsection{Cookies}
%
\subsection{Menü}
%

%EOF
