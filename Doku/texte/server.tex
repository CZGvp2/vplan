\section{Server}
\subsection{Funktionen}
Der Server der Webapplikation muss
mehrere Aufgaben bewältigen. Ein Webserver muss \texttt{HTTP} bedienen können, um
dem Browser des Nutzers eine \HTML-Oberfläche bieten zu können. Dazu müssen
Vetretungsplan-Daten auf dem Server gepeichert, abgerufen, und bearbeitet
werden. Weiterhin muss der Server in der Lage sein, vom Nutzer hochgeladene
Dateien zu empfangen, konvertieren und zu seinen gepeicherten Daten
hinzuzufügen. Damit nur autorisierte Nutzer Daten bearbeiten können, ist eine
sichere Authenifizierung über eine Passworteingabe des Nutzers erforderlich.
Diese erforderliche Web-Sicherheit beinhält Hashwerte und die Verwendung von
\texttt{SSL}.

\subsection{Struktur}
Der Server baut auf den schon von Pyramid gegebenen Strukturen auf. In dem Webframework
sind schon viele Teile eines Servers gekapselt, sodass wir uns nicht direkt damit
außeinandersetzen mussten. Dies umfasst Authentifikation, also das Erkennen und Speichern
von Nutzern, Autorisierung, die Einschränkung aller Nutzer auf ihre Erlaubnisbereiche.
Weiterhin nimmt uns das Framework das gesamte Kommunikationsprotokolle \texttt{HTTP} ab.
Somit läuft die gesamte Kommunikation mit dem Client über Pyramid ab. Dies umfasst das Senden
von Daten via \texttt{GET}-Requests, das Empfangen von Daten durch \texttt{POST}-Requests
und die Kommunikation via \texttt{AJaX}.\\
Am Back End des Servers befinden sich die Python-Scripts. Diese haben Zugriff auf die auf dem
Server gespeicherten Daten, und steuern das Framework. Die gesamte applikationsspezifische
Programmlogik wird innerhalb dieser Scripts ausgeführt. Die Scripts bekommen Nutzerdaten von
Pyramid in Form Requests (Anfragen) vom Nutzer.
Diese verarbeiten diese je nach Art des Requests, bearbeiten gegebenenfalls die
gespeicherten Daten auf dem Server, und geben dem Framework alle Informationen für eine Antwort
an den Client. Meist bestehen diese Informationen aus einzelnen Variablen. \\Um dem Nutzer jedoch
eine Antwort zu senden, müssen diese Daten noch in ein dem Nutzer verständlichen Format wie
\texttt{HTML} gebracht werden. Diese Aufgabe wird von sogenannten Renderern erledigt. Renderer 
setzen aus einer Template (Vorlage), und den speziefischen Daten eine vollständige Antwort
zusammen. Für jeden Datensatz von den Python-Scripts, welcher an den Client gesendet werden
soll, gibt es eine bestimmte Template. So können zum Beispiel mit Leichtigkeit große,
komplexe HTML-Seiten generieren, indem man unveränderliche Bestandteile in Templates auslagert,
und nur die veränderlichen Daten der Scripts vom Renderer einsetzen zu lassen. Handelt es
sich nur um unveränderlichen Daten in einer Template, wenn es sich beispielsweise um ein reines
\texttt{CSS}-Stylesheet handelt, so ist kein Renderer von nöten. Dies wird durch sogenannte
Static Assets realisiert, welche ganze Dateien unbearbeitet dem Nutzer zur Verfügung stellen.

% TODO Request

\subsection{Web-Sicherheit}

\subsection{Bearbeiten von Daten}

\paragraph{Ablaufprotokoll}

\paragraph{Konvertieren der Daten}

\subsection{Abrufen von Daten}

\subsection{Fehlerbehandlung}
