\section{Server}
\subsection{Funktionen}
Der Server der Webapplikation muss
mehrere Aufgaben bewältigen. Ein Webserver muss \texttt{HTTP} bedienen können, um
dem Browser des Nutzers eine \HTML-Oberfläche bieten zu können. Dazu müssen
Vetretungsplan-Daten auf dem Server gepeichert, abgerufen, und bearbeitet
werden. Weiterhin muss der Server in der Lage sein, vom Nutzer hochgeladene
Dateien zu empfangen, konvertieren und zu seinen gepeicherten Daten
hinzuzufügen. Damit nur autorisierte Nutzer Daten bearbeiten können, ist eine
sichere Authenifizierung über eine Passworteingabe des Nutzers erforderlich.
Diese erforderliche Web-Sicherheit beinhält Hashwerte und die Verwendung von
\texttt{SSL}.


\subsection{Struktur}

\subsection{Web-Sicherheit}

\subsection{Bearbeiten von Daten}

\paragraph{Ablaufprotokoll}

\paragraph{Konvertieren der Daten}

\subsection{Abrufen von Daten}

\subsection{Fehlerbehandlung}
