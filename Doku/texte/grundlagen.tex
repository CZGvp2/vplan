\chapter{Theoretische Grundlagen}
\section{Fußnote}
\blindtext\footnote{Völlig sinnloser Text für eine Fußnote}

\blindtext
\section{Tabellen}

\begin{table}[ht]
 \caption{Beispieltabelle}
 \label{tab:beispiel}
 \centering
 \begin{tabular}{lcr}\toprule
   Kopf & der & Tabelle \\\midrule
   eins zwei drei & zwei drei vier & drei vier fünf \\
   vier fünf sechs& fünf sechs sieben& sechs sieben acht\\
   sieben acht neun& acht neun zehn & neun zehne elf \\
   zwölf & dreizehn & vierzehn \\\bottomrule
 \end{tabular}
\end{table}

\blindtext\footnote{Und hier gleich nochmal ein völlig sinnloser Text für eine Fußnote, der über mehr als zwei Zeilen fließt}

\chapter{Lösungsidee}
\section{Abbildungen}
\blindtext

\begin{figure}[ht]
 \centering
 \includegraphics[scale=.5]{bilder/LogoCZG.pdf}
 \caption{Beispielabbildung}
 \label{abb:beispiel}
\end{figure}

\blindtext(Siehe Abbildung \ref{abb:beispiel})

\section{Zitate}
\begin{quote}
 "`\blindtext"'\footnote{Heinemann, 1998, S.\,280}
\end{quote}
